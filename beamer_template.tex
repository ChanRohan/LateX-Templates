\documentclass{beamer}
%\documentclass[mathserif,11pt]{beamer}

%\usetheme{AnnArbor}
%\usetheme{Antibes}
%\usetheme{Bergen}
%\usetheme{Berkeley}
%\usetheme{Berlin}
%\usetheme{Boadilla}
%\usetheme{boxes}
%\usetheme{CambridgeUS}
%\usetheme{Copenhagen}
%\usetheme{Darmstadt}
%\usetheme{default}
%\usetheme{Frankfurt}
%\usetheme{Goettingen}
%\usetheme{Hannover}
%\usetheme{Ilmenau}
%\usetheme{JuanLesPins}
%\usetheme{Luebeck}
\usetheme{Madrid}
%\usetheme{Malmoe}
%\usetheme{Marburg}
%\usetheme{Montpellier}
%\usetheme{PaloAlto}
%\usetheme{Pittsburgh}
%\usetheme{Rochester}
%\usetheme{Singapore}
%\usetheme{Szeged}
%\usetheme{Warsaw}

\mode<presentation>
{
%\setbeamercolor*{palette primary}{use=structure,fg=white,bg=blue}
\setbeamercolor*{palette secondary}{use=structure,fg=white,bg=red}
\setbeamercolor*{palette tertiary}{use=structure,fg=white,bg=black}
\setbeamercolor{frametitle}{bg=blue}
}



%%%%%%%%%%%%%%%%%%%%%%%%%%%%%%%%%%%%%%%%%%%%%%%%%%%%%%%%%%%%%%%%%%%%%%%%%%%%%%%%%%%%%%%%%%%%%%

\usepackage{fontawesome}
\usepackage{xcolor}
\usepackage{texpower}
\usepackage{graphicx}
\usepackage{framed,color}
\usepackage{tikz}

%%%%%%%%%%%%%%%%%%%%%%%%%%%%%%%%%%%%%%%%%%%%%%%%%%%%%%%%%%%%%%%%%%%%%%%%%%%%%%%%%%%%%%%%%%%%%%%

\usecolortheme{dolphin}
%COLOR THEME----albatross beaver beetle crane dolphin dove fly lily orchid seahorse whale wolverine
%\usecolortheme[RGB={100,90,150}]{structure}		% where 0 ≤ † ≤ 255.

%\usefonttheme{structureitalicserif}
%FONT THEME------professionalfonts 	structureitalicserif 	structurebold 	structuresmallcapsserif	
%\setbeamertemplate{background canvas}{\includegraphics[width=
%\paperwidth,height=\paperheight]{../../home/rchand/cosmos.jpg}}

\setbeamertemplate{items}[ball] % traingle, circle. For the items in the frame
\setbeamertemplate{navigation symbols}{} %navigation symbols that appear in bottom can be removed by using the command

\definecolor{shadecolor}{rgb}{0.5,0.5,0.5}  % shadecolcor for shaded  text block

% customBlock : Custom framed text box new environment
\newenvironment<>{customBlock}[1]{%
  \begin{actionenv}#2%
      \def\insertblocktitle{#1}%
      \par%
      \mode<presentation>{%
        \setbeamercolor{block title}{fg=white,bg=blue}
       \setbeamercolor{block body}{fg=black,bg=olive!50}
       \setbeamercolor{itemize item}{fg=orange!20!black}
       \setbeamertemplate{itemize item}[triangle]
     }%
      \usebeamertemplate{block begin}}
    {\par\usebeamertemplate{block end}\end{actionenv}}

% For mdframed the packages is used. 
\usepackage{tikz}
\usepackage[framemethod=tikz]{mdframed}

% Transparent text box with tikz is defiend here
\newmdenv[tikzsetting={draw=black,fill=white,fill opacity=0.7, line width=4pt},backgroundcolor=none,leftmargin=0,rightmargin=0,innertopmargin=4pt,skipbelow=\baselineskip,%
skipabove=\baselineskip]{TitleBox}

%%%%%%%%%%%%%%%%%%%%%%%%%%%%%%%%%%%%%%%%%%%%%%%%%%%%%%%%%%%%%%%%%%%%%%%%%%%%%%%%%%%%%%%%%%%%%%%

\title{Title}
\subtitle{Subtitle}
\author{Rohan Chandrakar}
\date{\today}
\institute[] { Center for Basic Sciences, \\Pt. Ravishsnakar Shukla University Raipur}
\subject{Subject} %This is only inserted into the PDF information catalog. Can be left out. 


%%%%%%%%%%%%%%%%%%%%%%%%%%%%%%%%%%%%%%%%%%%%%%%%%%%%%%%%%%%%%%%%%%%%%%%%%%%%%%%%%%%%%%%%%%%%%%%
%%%%%%%%%%%%%%%%%%%%%%%%%%%%%%%%%%%%%%%%%%%%%%%%%%%%%%%%%%%%%%%%%%%%%%%%%%%%%%%%%%%%%%%%%%%%%%%
%%%%%%%%%%%%%%%%%%%%%%%%%%%%%%%%%%%%%%%%%%%%%%%%%%%%%%%%%%%%%%%%%%%%%%%%%%%%%%%%%%%%%%%%%%%%%%%
\begin{document}

% Black Frame at the first slide
\bgroup
\setbeamercolor{background canvas}{bg=black}
\begin{frame}[plain]{}
\end{frame}
\egroup

{
 \usebackgroundtemplate{\includegraphics[width=1.0\paperwidth]{/home/rchand/cosmos.jpg}}
  \begin{frame}[plain] 

  \begin{TitleBox}
  \maketitle  
    \end{TitleBox}
	\begin{flushright}
	\vspace{5.5em}
	\textcolor{white}{\tiny Image credit : Andrew Pontzen/Fabio Governato}
  \end{flushright}
  \end{frame}
}

%%%%%%%%%%%%%%%%%%%%%%%%%%%%%%%%%%%%%%%%%
\begin{frame}{Table of Contents}
\tableofcontents
\end{frame}

%%%%%%%%%%%%%%%%%%%%%%%%%%%%%%%%%%%%%%%%%%


\section{First Section}

\subsection{First Subsection}
\begin{frame}{First Subsection}
\begin{itemize}\itemsep7pt
\item {First Item at once}
\item {Second Item at once}\\
\item {Third Item at once}\\

Statement without Item
\end{itemize}
\end{frame}

\subsection{Second Subsection}
\begin{frame}{Second Subsection}
\begin{itemize}
\item {First \\ Part \\ of \\ the\\ frame}\pause   %reveals the parts of a slide one at a time
\item {Second \\ Part \\ of \\ the \\ frame}
\end{itemize}
\end{frame}

\subsection{Third Subsection}
\begin{frame}{Item With Enumeration}
\begin{enumerate}
\item {First Item }
\item {Second Item}\\
\item {Third Item}\\
Statement without Item
\end{enumerate}
\end{frame}

\begin{frame}{Customize Item}
\begin{description}\itemsep18pt
\item[FIRST] Study of life.
\item[SECOND] Science of matter and its motion.
\item[THIRD] Scientific study of mental processes and behaviour.

\end{description}
\end{frame}



\section{Second Section}

\begin{frame}{Second Section}
\begin{itemize}[<+->] %Item will appear one by one in sequence
\item {First Item one by one}
\item {Second Item one by one}
\item {Third Item one by one} \\ \uncover<4-> {This will come latter  }
\end{itemize}
\end{frame}

\section{Colored Text}
\begin{frame}{Colored Text}
		%DEFINED COLOR----------red green blue yellow orange purple violet magenta cyan brown black white darkgray lightgray gray
 	\begin{itemize}
 	
		\item        \textcolor{red}{This is red colored text}\\
        \item		\colorbox{black}{\textcolor{green}{This is Green text with black background }}\\
        \item		\fcolorbox{black}{blue}{This is black colored frame with blue background}\\
        %\definecolor{colour name}{rgb}{?, ?, ?}   where 0 ≤ ? ≤ 1
		%\definecolor{colour name}{RGB}{†, †, †}   where 0 ≤ † ≤ 255.
		       \definecolor{MyPurple}{RGB}{200, 0, 230}
				\definecolor{MyLightBlue}{rgb}{0.3, 0.6, .7}
		\item		\textcolor{MyPurple}{This is my defined purple color} \\
		\item		\textcolor{MyLightBlue}{This is my defined light blue}
	
 	\end{itemize}
\end{frame}
%\definecolor{colour name}{rgb}{?, ?, ?}   where 0 ≤ ? ≤ 1
%\definecolor{colour name}{RGB}{†, †, †}   where 0 ≤ † ≤ 255.

%%%%%%%%%%%%%%%%%%%%%%%%%%%%%%%%%%%%%%%%%%%%%%%%%%%%%%%%%%%%%%%%%%%%%%%%%%%%%%%%%%%%%%%%%%%%%%%%%%%%%%%%%%%%%%%%%%%%%%%%%%%%%%%%%%
\section{Image Section}
\subsection{Only Image}
\begin{frame}{Images in LATEX}
\begin{figure}%[hbtp]
%\centering
%\includegraphics[scale=1]{../CMB.jpg}
%\includegraphics[height=\textheight]{../CMB.jpg}
\includegraphics[height=0.65\textheight]{../CMB.jpg}
\caption{Image name}
\end{figure}
\end{frame}


\subsection{Image with column}
\begin{frame}{Image with column} 

\begin{columns}

\column{.5\textwidth}

This is the first column\\

\includegraphics[height=0.55\textheight]{../CMB.jpg}


\column{.5\textwidth}

This is the Second column \\
\begin{figure}
\includegraphics[height=0.65\textheight]{../CMB.jpg}
\caption{Image name}
\end{figure}

\end{columns}
\end{frame}

%%%%%%%%%%%%%%%%%%%%%%%%%%%%%%%%%%%%%%%%%%%%%%%%%%%%%%%%%%%%%%%%%%%%%%%%%%%%%%%%%%%%%%%%%%%%%%%%%%%%%%%%%%%%%%%%%%%%5

\section{Framed Text and Custom Block}
%New environment has been made for different type of text box with addition to background image

%   1. Shaded Text Box  : shade color is defined in preamble
{
  \usebackgroundtemplate{\includegraphics[width=1.0\paperwidth]{/home/rchand/cosmos.jpg}}

	\begin{frame}{Framed Text and Custom Block}
		\begin{shaded}
			Different types of framed text. \\
			  This one is a block made with frame package
		\end{shaded}

%	2. Custom block with the new environment defined in preamble.
		\begin{customBlock}{Title of customized  block}
		    Content of textbox \\
		    More content
	    \end{customBlock}
	     
%	3. mdframed text box locally defiined here
		 \begin{mdframed}[outerlinewidth=3,leftmargin=0,%
		 rightmargin=20,backgroundcolor=white,%
outerlinecolor=black,innertopmargin=2pt,%
splittopskip=\topskip,skipbelow=\baselineskip,%
skipabove=\baselineskip]
  This is another custom frame made with mdframed, the options set locally here
  \end{mdframed}	
			
%	4. 	This textbox is added with the transperancy with tikz
		\begin{TitleBox}
		  A custom frame made with new md environment (via mdframed), and transparnecy via tikz
		  \end{TitleBox}

	\end{frame}

















\end{document}
